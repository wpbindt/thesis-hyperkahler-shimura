\chapter*{Summary}
In this thesis I study to what extent the moduli stacks of polarized hyperk\"ahler varieties (for example, K3 surfaces) are related to Shimura stacks. I focus in particular on hyperk\"ahler varieties defined over non-closed fields, and the ramifications of Deligne's reciprocity law for such varieties. Chapters~1 and~2 serve as introductions to Shimura stacks and moduli of polarized hyperk\"ahler varieties, respectively, and can be read independently of each other. From Chapter~3 on, every chapter depends on all chapters preceding it.

Chapter~1 is an introductory chapter, and gives a detailed global overview of the thesis and the main results.

The second chapter is an expository chapter about Shimura varieties and motives. The main result is that the canoncial model of a Shimura variety of abelian type (for example, an orthogonal Shimura variety) is a moduli space of abelian motives. This result is due to Deligne and Milne, and this chapter is a more Tannakian exposition of their work.

Chapter~3 gives an introduction to polarized hyperk\"ahler varieties and their moduli. The main result, which is well known to the experts, is that the moduli stack of polarized hyperk\"ahler varieties is a separated Deligne-Mumford stack over $\QQ$.

In Chapter~4 I study the period map for hyperk\"ahler varieties. This is a morphism from a degree~2 \'etale covering of the moduli stack of complex polarized hyperk\"ahler varieties to an orthogonal Shimura stack. One of the main results of this chapter is that this map descends to a morphism over $\QQ$. This is proved by combining the results of Chapter~2 and~3 with Andr\'e's result that the motive of a hyperk\"ahler variety is abelian. Furthermore, the chapter contains stronger versions of this main result for two specific families of hyperk\"ahler varieties, namely K3 surfaces and hyperk\"ahler varieties of type $\KKKKK^{[n]}$.

The final chapter applies the results of Chapter~4 to obtain more concrete results about K3 surfaces, namely a computation of the spinor norm of monodromy operators on the second cohomology. The proof makes use of a result Deligne on the connected components of the canonical model of a Shimura variety, of which the chapter contains an exposition. This is then used to sharpen a result of Elsenhans and Jahnel on K3 surfaces over finite fields, and to give a non-trivial necessary condition for a lattice to be the N\'eron-Severi lattice of a K3 surface over a non-closed field.
