\documentclass[b5paper,10pt]{letter}

\usepackage[margin=2cm]{geometry}
%\usepackage[none]{hyphenat}
\usepackage{enumitem}% http://ctan.org/pkg/enumitem
\usepackage{calc}% http://ctan.org/pkg/calc

\begin{document}
\thispagestyle{empty}
Forensic examination of fire debris is a notoriously difficult analytical task due to the complexity and variability of samples encountered. The development of increasingly sophisticated analytical instrumentation facilitates greater sensitivity while drastically increasing the abundance of data produced in a single analytical run. Hybrid methods combining chemometrics and forensic statistics have been developed in order to make optimal use of the complex analytical results.\\

This dissertation addresses the interpretation of such measurements by considering competing hypotheses and using appropriate forensic reference data. Both detection and classification tasks for ignitable liquid residue in fire debris samples are explored as valuable contributions to the investigative process and potential use in a judicial context. Assessing the evidential value of the measurements requires understanding of the analytical signal in addition to extensive comprehension of the many sources of variation within and between samples.\\

Preliminary data processing of chromatographic samples is explored including a discussion of both common practices and an exposition of novel techniques. The statistical modelling of chemically distinctive classes of ignitable liquids and the identification of their potentially discriminative characteristics is also presented. The methods and results presented herein emphasize the importance of a deep understanding of the data produced by analytical methods and a rigorous assessment of the statistical and computational assumptions made when articulating their evidential value. Statistical interpretation of fire debris analyses is based on a general probabilistic (likelihood ratio) framework for interpretation and evaluation of evidence.

\end{document}
